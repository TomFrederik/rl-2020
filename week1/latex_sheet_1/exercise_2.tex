\documentclass{article}
\usepackage[utf8]{inputenc}
\usepackage{amsmath}
\usepackage{amssymb}
\usepackage{amstext}
\usepackage{amsfonts}
\usepackage{enumitem}
\usepackage{scrextend}
\usepackage{mathrsfs}
\usepackage{titlesec}
\usepackage{titling}
\usepackage{float}
\usepackage{comment}
\usepackage{my_math}

\titleformat*{\section}{\normalsize\bfseries}
\setlength{\droptitle}{-3cm}

\title{Reinforcement Learning \\[3mm] Homework Chapters 1, 2}
\author{Tom Lieberum - ID 13253042 \and Erik Jenner - ID 13237896}

\begin{document}
\date{}
\maketitle


\section*{2.1: Dynamic Programming}

\begin{enumerate}[label*=\arabic*.]

\item

Stochastic:
\begin{align*}
v^\pi(s) \,&=\, \expval{a\sim\pi(\cdot|s)}{q^\pi(s,a)}\\[3mm]
\,&=\, 
\begin{cases}
\sum_{a\in\cal A}\pi(a|s) \cdot q^\pi(s,a), \,\, \text{discrete case}\\[2mm]
\int_{\cal A}\pi(a|s) \cdot q^\pi(s,a) \,da, \,\, \text{continuous case}
\end{cases}
\end{align*}


\noindent Deterministic:
\[
v^\pi(s) \,=\,  q^\pi(s,\pi(s))
\]

\item
\begin{align*}
q_{k+1}(s,a) &= \expval{}{R_{t+1} + \gamma \max_{a'\in\cal A} {q_k(S_{t+1},a')} \,\big\rvert\, S_t=s, A_t=a}\\[3mm]
&= \sum_{s',r} p(s',r|s,a)\left[ r + \gamma\, \max_{a' \in \cal A}\,q_{k}(s',a') \right]
\end{align*}


\item
For all states $s$ and actions $a$:

\[
Q^\pi(s,a) \,\leftarrow\, \sum_{s',r}p(s',r|s,a) \left[ r + \gamma\, \sum_{a'\in\cal A} \pi(a'|s')\cdot Q^\pi(s',a') \right]
\]
For deterministic policies $\pi(s)$ we get

\[
Q^\pi(s,a) \,\leftarrow\, \sum_{s',r}p(s',r|s,a) \left[ r + \gamma\, \cdot Q^\pi(s',\pi(s')) \right]
\]

\item
\[
\pi(s) \leftarrow \argmax_{a\in \cal A}Q^{\pi}(s,a)
\]

\end{enumerate}

\section*{2.2: Coding Assignment - Dynamic Programming}
\begin{enumerate}[label*=\arabic*.]
\item
Cf. code.

\item

During each iteration of value iteration, for every state we compute the maximum over $|\cal A|$ elements which take themselves $O(|\cal S||\cal R|)$ operations to compute. So each iteration in VI has time complexity $O(|\cal S|^2|\cal R||\cal A|)$, if we assume that evaluating $v$ has constant time complexity (e.g. when we are in a tabular case).

In policy iteration, we have two steps per iteration.
In the evaluation step, each improvement in the evaluation costs $O( |\cal S||\cal R||\cal A|)$ time for each state, which we need to multiply by the number of iterations it takes for the evaluation to converge to $v^\pi$. So if it takes $O(N)$ iterations to converge, then this step has time complexity $O(N|\cal S|^2|\cal R||\cal A|)$

In the improvement step, we essentially perform the same computation as in the value iteration update, just that we are concerned with the argmax, rather than the max. So this step costs $O(|\cal S|^2|\cal R||\cal A|)$ as well.

So, if we assume that it takes value iteration a factor of $\alpha$ more global iterations to converge than policy iteration (e.g. $\alpha = 1$), because it is working with a less accurate estimate of the current value function, then value iteration is faster by a factor of $N/\alpha$.

\end{enumerate}

















\begin{comment}
\section*{2.3: Exam Question: Dynamic Programming}
\begin{enumerate}[label*=\arabic*.]

\item

\begin{enumerate}[label=(\alph*)]
\item False, both converge to the optimal policy.


\item
Technically False. Value Iteration does not evaluate any policy, it just improves the estimate of the optimal value function in each step, given the current estimate. In fact, intermediate value functions may not correspond to any particular policy.

\end{enumerate}

\item
When policy iteration stabilizes, then the optimal actions do not change. Say the stable policy is $\pi$. For $v^\pi$ we always have the Bellman expectation equation for a deterministic policy $\pi$ (which we can assume because the improvement step creates a deterministic policy):
\[ 
v^\pi \,=\, \sum_{s', r} p(s',r|s,\pi(s)) \left[ r + \gamma v^\pi(s') \right] 
\]

Since $\pi$ is stable and $\pi$ is chosen such to maximize the RHS above we can also replace it with a max over actions.

It remains to be shown that the solution to the equation above is unique and thereby $v^*=v^\pi$.
\end{enumerate}

\section*{2.4: Monte Carlo}
\begin{enumerate}[label*=\arabic*.]

\item

\begin{enumerate}[label=(\alph*)]
\item 
\[
v(s_0)\,=\,0.9^2 + 0.9^4 + 0.9^3 \,=\, 2.1951
\]

\item
\[
v(s_0)
\]

\end{enumerate}
\end{enumerate}
\end{comment}

\end{document}